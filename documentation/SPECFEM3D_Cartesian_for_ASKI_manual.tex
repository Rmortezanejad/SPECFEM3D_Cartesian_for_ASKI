%-----------------------------------------------------------------------------
%   Copyright 2013 Florian Schumacher
%
%   This file is part of the SPECFEM3D_Cartesian_for_ASKI manual as a LaTeX 
%   document with main file SPECFEM3D_Cartesian_for_ASKI_manual.tex
%
%   Permission is granted to copy, distribute and/or modify this document
%   under the terms of the GNU Free Documentation License, Version 1.3
%   or any later version published by the Free Software Foundation;
%   with no Invariant Sections, no Front-Cover Texts, and no Back-Cover Texts.
%   A copy of the license is included in the section entitled ``GNU
%   Free Documentation License''. 
%-----------------------------------------------------------------------------
%
\documentclass[12pt,a4paper]{article}

\usepackage[english]{babel} %language selection
\selectlanguage{english}

\pagenumbering{arabic}

\usepackage[affil-it]{authblk}
\usepackage{times} % 'times new roman' script style

%\usepackage{amsmath}
%\usepackage{amssymb}
%\usepackage{graphicx}

% use package url with [obeyspaces] in order to correctly display \nolinkurl WITH spaces 
%(used in \newcommand{\lcode} below). As hyperref internally loads package url, you can pass
% option obeyspaces of package url to package hyperref as follows
\PassOptionsToPackage{obeyspaces}{url}\usepackage{hyperref}
%\hypersetup{colorlinks, 
%           citecolor=black,
%           filecolor=black,
%           linkcolor=black,
%           urlcolor=black,
%           bookmarksopen=true,
%           pdftex}
%\hfuzz = .6pt % avoid black boxes

% the following is an ugly solution of allowing line breaks in urls additionally after every normal 
% alphabetic character which (if \nolinkurl is used in \newcommand{\lcode} below) at all allows line 
% breaks of long routine names like 'transformToStandardCellInversionGrid', BUT of course also breaks
% any other term formatted by \lcode at any character, which is maybe not very nice.
\let\origUrlBreaks\UrlBreaks
%\renewcommand*{\UrlBreaks}{\origUrlBreaks\do\a\do\b\do\c\do\d\do\e\do\f\do\g\do\h\do\i\do\j\do\k\do\l\do\m\do\n\do\o\do\p\do\q\do\r\do\s\do\t\do\u\do\v\do\w\do\x\do\y\do\z\do\A\do\B\do\C\do\D\do\E\do\F\do\G\do\H\do\I\do\J\do\K\do\L\do\M\do\N\do\O\do\P\do\Q\do\R\do\S\do\T\do\U\do\V\do\W\do\X\do\Y\do\Z}


%% POSSIBLE PACKAGES TO DISPLAY CODE
%%
%% package alltt: verbatim environment within which math is displayed correctly
%% usage: \begin{alltt}\end{alltt}
%\usepackage{alltt}
%%
%% package listings: provides environments to display code fragments (with a lot of special characters) in a more evolved fashion than verbatim (alltt)
%% only uncomment (both next lines), if used in \newcommand{\lcode} below
%\usepackage{listings}
%\lstset{basicstyle =\ttfamily}%\small}

\usepackage[paperwidth=21.0cm,paperheight=29.7cm, left=2.5cm,right=2.5cm,top=2.0cm,
            bottom=2.0cm,headheight=0in,footskip=1.0cm]{geometry}
%-------------------------------
%
% COMMANDS FOR IN-LINE PHRASES IN CODE-STYLE
%
%%% ttfamily does not properly support any special characters
%\newcommand{\lcode}[1]{ {\ttfamily #1 }}
%
%%% lstinline is a good solution, in general, but it makes problems in line breaks!
%\newcommand{\lcode}[1]{\lstinline[breaklines=true]$#1$}
%
%%% although there are no actual links, it uses the same font as lstinline (when \lstset{basicstyle =\ttfamily}), 
%%% but produces better line breaks!
\newcommand{\lcode}[1]{\nolinkurl{#1}}
%
%%% need \lcodetitle, since \nolinkurl in a title of a numerated (sub)section (not *) causes problems in bookmark 
%%% view in adobe reader (why?! what is the actual problem?), \lcodetitle, however, does NOT support stuff like '_' etc.
\newcommand{\lcodetitle}[1]{ {\ttfamily #1} }
%
%
\newcommand{\ASKI}{ {\ttfamily ASKI} }
%
%
% OTHER NEW COMMANDS
%
\newcommand{\inotice}[1]{ \fbox{\parbox[t]{0.9\textwidth}{{\bf Important:} \\#1}} }
\newcommand{\notice}[1]{ \fbox{\parbox[t]{0.9\textwidth}{#1}} }
\newcommand{\myref}[1]{\ref{#1} (page~\pageref{#1})}
\newcommand{\myaref}[1]{$\rightarrow$~\ref{#1} (page~\pageref{#1})}
%
%-------------------------------
%
% END OF PREAMBLE
%####################################################################
%
\begin{document}
%
\setlength{\parindent}{0cm}
\addtolength{\parskip}{0.1cm}
% TeX’s first attempt at breaking lines is performed without even trying hyphenation: 
% TeX sets its “tolerance” of line breaking oddities to the internal value \pretolerance
% an “infinite” tolerance is represented by the value 10000, but may lead to very bad line breaks indeed!
%\pretolerance=10000
%
%-------------------------------
% TITLE PAGE
%
% without \usepackage[affil-it]{authblk} e.g.:
%\author{Florian Schumacher \thanks{\texttt{florian.schumacher@rub.de}; corresponding author} \and Wolfgang Friederich \thanks{\texttt{wolfgang.friederich@rub.de}}}
%
\title{Using {\tt \Huge SPECFEM3D\_Cartesian-2.1} for \\ \tt {\Huge ASKI} {\rm--} {\Huge A}{\large nalysis of} {\Huge S}{\large ensitivity \\ and} {\Huge\tt K}{\large ernel} {\Huge\tt I}{\large nversion, version 0.3} }
%\author[1]{Florian Schumacher \thanks{\texttt{florian.schumacher@email.address}; corresponding author}}
\author[1]{Florian Schumacher}
\author[1]{Wolfgang Friederich}
\affil[1]{Ruhr-Universit\"at Bochum} % for this you need \usepackage[affil-it]{authblk}
\date{\today}
%\date{6.12.2004}
%\date{} % no date
\maketitle
%
%-------------------------------
% LICENSE
Copyright \copyright 2013 Florian Schumacher.
Permission is granted to copy, distribute and/or modify this document
under the terms of the GNU Free Documentation License, Version 1.3
or any later version published by the Free Software Foundation;
with no Invariant Sections, no Front-Cover Texts, and no Back-Cover Texts.
A copy of the license is included in the section entitled ``GNU
Free Documentation License''.

\vspace{1cm}

This documentation was written in the hope that it will be useful to the user,
but it \emph{cannot be assured} that it is accurate in every respect or complete in any sense.\\
Please do not hesitate to improve this documentation by incorporating your experiences with 
\lcode{SPECFEM3D for ASKI} and your personal experience of getting used to it. 

Furthermore, my moderate experience with \LaTeX may well give rise to improving the document 
style, hence the readability of the manual as a whole, as well as the coding style of the 
\lcode{.tex} file. 

The \LaTeX source files and all related components of this document are contained in the 
\lcode{SPECFEM3D_Cartesian-2.1 for ASKI 0.3} package, available via \url{http://www.rub.de/ASKI}
\begin{flushright}
Florian Schumacher, 2013
\end{flushright}
%
%-------------------------------
% SECTION Introduction
%#############################################################
\section*{Guide Through This Manual}
%#############################################################
%
We assume that you have sufficient knowledge of how to run the regular \lcode{SPECFEM3D_Cartesian} code.

For details on how to extend the regularly installed \lcode{SPECFEM3D_Cartesian-2.1} code to produce
output for \ASKI, please read section~\ref{extent_to_ASKI}.

Before you start using the code to produce output for \ASKI, please consider the general
comments in section~\ref{general_stuff}.

If you are going to use the automated python script \\ \lcode{run_specfem3dCartesianForASKI_simulations.py} 
then please start reading section~\ref{use_script}. 

If you want to conduct one single simulation producing some output for \ASKI, please start 
reading section~\ref{no_script}.

Section~\ref{file_Par_file_ASKI} is intened to be used as a reference section only.

Bracketed comments starting with ``{\bf TODO IN THE FUTURE:}'' are intended to mark ideas for future work. 
So please ignore if you are just applying the code.
%
%-------------------------------
% TABLE OF CONTENTS
\newpage
\tableofcontents
\newpage
%
%-------------------------------
% SECTION General things
%#############################################################
\section{General Stuff to Consider} \label{general_stuff}
%#############################################################
%
\begin{itemize}
\item parameters \lcode{FILE_KERNEL_REFERENCE_MODEL} and \lcode{FILE_WAVEFIELD_POINTS} 
  of the parameter file for a specific iteration step must be set to some main \ASKI output file,
  which is the basefile name of \lcode{ASKI_outfile} extendet by \lcode{.main}, see \ref{Par_file_ASKI,sub:output}.
  Use the main \ASKI output file of some arbitrary \ASKI output, e.g.\ the kernel displacement output of the first source.
\item As there is a fixed order assumed of \ASKI wavefield points (by procs and local element numbering), the 
  computation of many kernels (e.g.\ for many seismic paths in an inversion) can only be consistent, if the 
  \emph{same} mesh decomposition and the \emph{same} number of procs is used at all times 
  (for those kernels you want to use together, e.g.\ all kernels in your specific iteration step of an inversion).
  It may, hence, be sensible to use the same MPI Databases for all \lcode{SPECFEM3D for ASKI} simulations (adjust your 
  script \lcode{process.sh} in such a way, that you do not always recompile and decompose MESH, but only call the solver
  again (with changed parameter files and source files).
\item there is (probably?!) no proper support for multiple sources in case of producing output for \ASKI.
\item Green tensor simulations are done using  value \lcode{ALT} in the station definition as the 
  \lcode{FORCESOLUTION} \lcode{depth} value, in order to allow receivers to be located not only on the surface.
  It, hence, may be sensible to throughoutly use \lcode{USE_SOURCES_RECVS_Z = .true.} in \lcode{SPECFEM3D/src/shared/constants.h}.
\item As the coordinates of wavefield points (and, hence, inversion grid coordinates) \lcode{SPECFEM3D for ASKI} uses:\\
  First coordinate = \lcode{X}, second coordinate = \lcode{Y}, third coordinate = \lcode{Z}.
\item \dots
\end{itemize}
%
%-------------------------------
% SECTION single simulation without python script
%#############################################################
\section{One Single Simulation} \label{no_script}
%#############################################################
%
As usual, you need to do (external or internal) meshing for the appropriate (current) background model. 
See section \ref{import_model} for details on how to import the current model of an inverision (the model of the last
iteration step) into \lcode{SPECFEM3D}.

Set the regular \lcode{SPECFEM3D} files \lcode{Par_file}, \lcode{CMTSOLUTION / FORCESOLUTION} 
and \lcode{STATIONS} (standard \lcode{SPECFEM3D} functionality; only if you want to record any seismograms).

Additionally, you need to set file \lcode{Par_file_ASKI} to desired values. The file is described 
in detail in section~\ref{file_Par_file_ASKI}. 

After that, you are ready to run the code. As there is no \ASKI dependent change of \lcode{SPECFEM3D} 
components which are sensitive to compiling, you do not need to recompile the \lcode{SPECFEM3D} code every time you
run a \lcode{SPECFEM3D} simulation for \ASKI.
%
%-------------------------------
% SECTION using python script
%#############################################################
\section{Using Automated Python Script for Doing Several Simulations} \label{use_script}
%#############################################################
%
As usual, you need to do (external or internal) meshing for the appropriate (current) background model. 
See section \ref{import_model} for details on how to import the current model of an inverision (the model of the last
iteration step) into \lcode{SPECFEM3D}.

Python script \lcode{run_specfem3dCartesianForASKI_simulations.py} conducts the specified 
kernel simulations (as described inside the script on the top) by iteratively running 
\lcode{SPECFEM3D} simulations, using appropriate settings of the parameter files for each iteration. \\
You need to edit the first lines of the script (above definition of \lcode{class simulation}) and set 
all variables defined there to appropriate values, as described in the comments in the script 
({\bf TODO IN THE FUTURE:} maybe it is better to have an input (file?) mechanism to this script. But then: more
overhead/extra requirements (packages, arguments handling) to cope with on cluster machines \dots)

The python script may not be suitable for the HPC system you are using! If you are not able to adapt 
it in a way which makes it possible to be used, you might have to figure out an analogous way yourself 
how to perform the tasks done by this script.

In case of using the provided python script \lcode{run_specfem3dCartesianForASKI_simulations.py},
some parameters in \lcode{SPECFEM3D} files \lcode{CMTSOLUTION}, \lcode{FORCESOLUTION}, \lcode{Par_file} 
and in file \lcode{Par_file_ASKI} are automatically changed, while the script iteratively conducts 
\lcode{SPECFEM3D} simulations. \\
In the following, only those parameters/lines are listed, which, if necessary, need to be set 
manually before running this python script. All other parameters are set by the script.

%+++++++++++++++++++++++++++++++++++++++++++++++++++++++++++++++++++++++
\subsection{Manually Setting \lcodetitle{Par\_file\_ASKI}}
%+++++++++++++++++++++++++++++++++++++++++++++++++++++++++++++++++++++++
The following \lcode{Par_file_ASKI} parameters need to be set manually before running the python script, 
since they are not changed/set by the script.
\begin{itemize}
\item \lcode{OVERWRITE_ASKI_OUTPUT}
\item \lcode{ASKI_DFT_double}
\item \lcode{ASKI_DFT_apply_taper,ASKI_DFT_taper_percentage}
\item in case of \lcode{define_ASKI_output_volume_by_inversion_grid = False}, you need to manually set
  all parameters concerning the inversion grid, i.e.\ \lcode{ASKI_type_inversion_grid}, 
  \lcode{ASKI_(cw)(xyz)}, \lcode{ASKI_rot_(XYZ)}
\end{itemize}
%
%+++++++++++++++++++++++++++++++++++++++++++++++++++++++++++++++++++++++
\subsection{Manually Setting \lcodetitle{FORCESOLUTION}}
%+++++++++++++++++++++++++++++++++++++++++++++++++++++++++++++++++++++++
%
The python script \emph{always} automatically sets ``latitude:'', ``longitude:'', ``depth:'', 
``factor force source:'', ``component dir vect source E:'', ``component dir vect source N:'', 
``component dir vect source Z\_UP:''. In case of displ and gt simulations, additionally ``hdur:'' is set to ``0.''. 
So, if you wish to do a data simulation for single force sources, using a different ``hdur:'' value, you 
should conduct those in a separate run of the python script.
%
%+++++++++++++++++++++++++++++++++++++++++++++++++++++++++++++++++++++++
\subsection{Manually Setting \lcodetitle{CMTSOLUTION}}
%+++++++++++++++++++++++++++++++++++++++++++++++++++++++++++++++++++++++
%
The python script \emph{always} automatically sets ``latitude:'', ``longitude:'', ``depth:'', 
``Mrr:'', ``Mtt:'', ``Mpp:'', ``Mrt:'', ``Mrp:'', ``Mtp:''. In case of displ and simulations, additionally 
``half duration:'' is set to ``0.''. So, if you wish to do a data simulation for single force sources, 
using a different ``half duration:'' value, you should conduct those in a separate run of the python script.
%
%+++++++++++++++++++++++++++++++++++++++++++++++++++++++++++++++++++++++
\subsection{Manually Setting \lcodetitle{STATIONS}}
%+++++++++++++++++++++++++++++++++++++++++++++++++++++++++++++++++++++++
%
The standard \lcode{SPECFEM3D} \lcode{STATIONS} file should contain the definition of stations as
in the \ASKI file \lcode{file_stations_list}, in consistend \lcode{SPECFEM3D} notation, i.e.\ coordinate
columns being lat ( = Y, third column of \lcode{STATIONS} and fourth column of \lcode{file_stations_list}) and 
lon ( = X, fourth column of \lcode{STATIONS} and third column of \lcode{file_stations_list}) and 
elev ( = Z, sixth column of \lcode{STATIONS} and fifth column of \lcode{file_stations_list}). 

You must also assure to use the very same station names and network codes in file \lcode{STATIONS} as in 
\ASKI file \lcode{file_stations_list}!
%
%-------------------------------
% SECTION importing an external model
%#############################################################
\section{Importing the currently inverted model for next iteration step} \label{import_model}
%#############################################################
%
Exported \lcode{.kim} files (as produced by \ASKI program \lcode{exportKim}) may be read used as a model by 
\lcode{SPECFEM3D for ASKI}, as explained in the following:

The model values defined by the kernel inverted model will be \emph{superimposed} onto the \lcode{SPECFEM3D} 
default model values as defined by e.g.\ \lcode{Cubit}, using a special implementation of the \lcode{SPECFEM3D} 
module \lcode{model_external_values}.

Set \lcode{MODEL = external} in \lcode{Par_file}. Then directory \lcode{DATA} must contain a file
named \lcode{model_external_ASKI} which must contain 3 lines:
\begin{itemize}
\item[line 1] \emph{interpolation type:}\\
  The first line may be of the following formats:
  \begin{itemize}
  \item \lcode{shepard_standard}
    meaning that a standard inverse distance interpolation, respecting for direction of neighbours, after Shepard
    is used to interpolate model values from the inversion grid cell centers onto the new \lcode{SPECFEM3D} GLL points.
    The radii of the inversion grid cells are used to decide whether a GLL point is in range of an inversion grid cell
    or not.
  \item \lcode{shepard_factor_radius  FACTOR}
    whereby \lcode{FACTOR} is a factor to be applied to the inversion grid cell radii before applying the same
    interpolation method as in case \lcode{shepard_standard}.
  \end{itemize}
\item[line 2] \emph{model file type:}\\
  only \lcode{kim_export} supported by now, meaning that the file given in line 3 must be an exported \lcode{.kim} file
  as produced by \ASKI program \lcode{exportKim}.
\item[line 3] \emph{name of model file:}\\
  name of model file of type as defined in line 2, which is expected to be in directory \lcode{DATA}
\end{itemize}
%
%-------------------------------
% SECTION preparing synthetic data
%#############################################################
\section{Preparing Synthetic Data as Expected by \ASKI}
%#############################################################
%
Use binary \lcode{createSpecfem3dSyntheticData}. Calling \lcode{createSpecfem3dSyntheticData -h} will print a
help message how to use it.

It is assumed that a copy of the \lcode{OUTPUT_FILES} folder (without the \lcode{MPI_DATABASES} files etc...)
of all involved \lcode{SPECFEM3D} simulations (which contain the standard seismograms files) can be found at 
the path as choosen by the automated python script (see~\ref{use_script}), i.e.\ filename of the kernel displacement 
file for the respective event with the extension \lcode{_OUTPUT_FILES}. 
The synthetic data then is written in the required form to path \lcode{PATH_SYNTHETIC_DATA}, where the filenames are by
convention \lcode{synthetics_EVENTID_STATIONNAME}.  

The option \lcode{-data} allows you to also transform synthetically created ``real'' data (e.g.\ for a 
pure synthetic inversion) to the required form. Also in this case it is assumed that a copy of the 
\lcode{OUTPUT_FILES} folders can be found at paths as choosen by the automated python script, i.e.\ the base
filename of the measured data for the respective event with extension \lcode{_OUTPUT_FILES}. The ``real'' data 
then is written in the required form to the measured data files with base filenames as defined in the iteration step 
info database. 
%
%-------------------------------
% SECTION preparing measured data
%#############################################################
\section{Preparing synthetically computed ``measured'' data as expected by \lcodetitle{ASKI}}
%#############################################################
%
You can produce \ASKI files for measured data in the required form from \lcode{SPECFEM3D for ASKI} ``data'' 
simulations (e.g.\ produced by automated python script,~\ref{use_script}). This functionality may
be used for synthetic tests, in which you must produce data for some perturbed earth model, which 
is treated as measured data.

Use binary \lcode{createSpecfem3dMeasuredData}. Calling \lcode{createSpecfem3dMeasuredData -h} will print a
help message how to use it.

It is assumed that a copy of the content of the \lcode{OUTPUT_FILES} folder (without the \lcode{MPI_DATABASES} files etc...)
of the ``data'' simulations (which contain the standard seismograms files) can be found at 
directory \lcode{PATH_MEASURED_DATA/data_EVENTID_OUTPUT_FILES}. 
The measured data files then are written in the required form to path \lcode{PATH_MEASURED_DATA}, where the filenames are by
convention \lcode{data_EVENTID_STATIONNAME_COMP}.  
%
%-------------------------------
% SECTION Par_file_ASKI
%#############################################################
\section{File \lcodetitle{Par\_file\_ASKI}} \label{file_Par_file_ASKI}
%#############################################################
%
File \lcode{Par_file_ASKI} is, just like the file \lcode{Par_file}, located in directory 
\lcode{DATA/} of your current \lcode{SPECFEM3D} example. It basically controls \ASKI functionality 
\lcode{SPECFEM3D} if used along with an \ASKI extended \lcode{SPECFEM3D} installation. If in such an 
installation file \lcode{Par_file_ASKI} is not present, no \ASKI output is produced and 
\lcode{SPECFEM3D} runs with standard functionality. 

In the following, we give a short description of the functionality of parameters defined
in file \lcode{Par_file_ASKI}.
%+++++++++++++++++++++++++++++++++++++++++++++++++++++++++++++++++++++++
\subsection{\lcodetitle{ASKI} output} \label{Par_file_ASKI,sub:output}
%+++++++++++++++++++++++++++++++++++++++++++++++++++++++++++++++++++++++
%----------------------------------------------------------------------
\subsubsection*{\lcode{COMPUTE_ASKI_OUTPUT, OVERWRITE_ASKI_OUTPUT}}
%----------------------------------------------------------------------
Parameter \lcode{COMPUTE_ASKI_OUTPUT} controls if any ASKI functionality is applied by \lcode{SPECFEM3D} and
output files (i.e. kernel green tensor kernel displacement files) are produced. If true, \lcode{OVERWRITE_ASKI_OUTPUT}
controls if those files shall be overwritten if existend or not. If false and files exist, the \lcode{SPECFEM3D}
solver will terminate raising an error message.
%----------------------------------------------------------------------
\subsubsection*{\lcode{ASKI_outfile, ASKI_output_ID}}
%----------------------------------------------------------------------
\lcode{ASKI_outfile} defines the absolute base file name of \ASKI output files. \lcode{ASKI_output_ID} is a character
string of maximum lenght as defined by parameter \lcode{length_ASKI_output_ID} in file \lcode{specfem3D_par_ASKI.f90} 
with which all output files of a certain run will be taged, and it will be used to check consistency of the files
(could be a timestamp, eventID, station name, component etc).
%+++++++++++++++++++++++++++++++++++++++++++++++++++++++++++++++++++++++
\subsection{Frequency discretization}
%+++++++++++++++++++++++++++++++++++++++++++++++++++++++++++++++++++++++
The double precision \lcode{df} [Hz] and integer values \lcode{jf} have the following meaning:
The spectra are saved for all frequencies \lcode{f = (jf)*df} [Hz].
%----------------------------------------------------------------------
\subsubsection*{\lcode{ASKI_df, ASKI_nf, ASKI_jf}}
%----------------------------------------------------------------------
\lcode{ASKI_df} is a predefined frequency step that is used to evaluate the spectrum. In case we want to do 
an inverse FT in case of time-domain sensitivity kernel computation, we need to choose \lcode{ASKI_df} with care 
as \lcode{ASKI_df = 1/length_of_time_series} and suitably high frequency indices (dependent on frequency content).
Otherwise we could lose periodicity (if in \lcode{exp^(-i2pi(k)(n)/N)} \lcode{N} is no integer, these are no 
roots of 1 anymore). The spectra are saved for frequencies \lcode{f = (ASKI_jf)*ASKI_df} (\lcode{ASKI_nf} many).
%----------------------------------------------------------------------
\subsubsection*{\lcode{ASKI_DFT_double}}
%----------------------------------------------------------------------
Choose precision of Discrete Fourier Transform. If there is enough memory available, it is highly recommended
to use \lcode{ASKI_DFT_double = .true.} in which case double complex spectra are hold in memory (single precision is 
written to file, though, but less roundoffs during transformation). Otherwise choose \lcode{ASKI_DFT_double = .false.}
in which case single precision spectra will be used in memory. The transformation coefficients \lcode{exp^(-i*2pi*f*t)} 
are always in double complex precision!
%----------------------------------------------------------------------
\subsubsection*{\lcode{ASKI_DFT_apply_taper, ASKI_DFT_taper_percentage}}
%----------------------------------------------------------------------
Decide whether the (oversampled, noisy, ...) time series should be tapered by a hanning taper (on tail)
before (i.e.\ while) applying the discrete fourier transform (on-the-fly). If \lcode{ASKI_DFT_apply_taper = .true.},
the value of \lcode{ASKI_DFT_taper_percentage} (between 0.0 and 1.0) defines the amount of
total time for which the hanning taper will be applied at the tail of the time series.
%+++++++++++++++++++++++++++++++++++++++++++++++++++++++++++++++++++++++
\subsection{Inversion grid}
%+++++++++++++++++++++++++++++++++++++++++++++++++++++++++++++++++++++++
%----------------------------------------------------------------------
\subsubsection*{\lcode{ASKI_type_inversion_grid}}
%----------------------------------------------------------------------
ASKI supports several types of inversion grids for \lcode{FORWARD_METHOD = SPECFEM3D}.
\lcode{ASKI_type_inversion_grid = }
\begin{enumerate}
\item (\lcode{TYPE_INVERSION_GRID = ccsInversionGrid}) \\ 
  NOT TO BE USED WITH SPECFEM3D Cartesian! {\bf NOT SUPPORTED YET}\\
  ASKI internal, but SPECFEM independent spherical inverison grid
\item (\lcode{TYPE_INVERSION_GRID = scartInversionGrid})\\
  \ASKI internal, but \lcode{SPECFEM} independent cartesian inversion grid:\\
  The values for \ASKI output are stored at all inner GLL points of spectral elements which lie
  inside the block volume defined below by parameters \lcode{ASKI_(cw)(xyz)}.
  \ASKI loactes the coordinates of those points inside the inversion grid cells and computes
  integration weights for them.
\item (\lcode{TYPE_INVERSION_GRID = ecartInversionGrid}) \\
  External inversion grid provided e.g.\ by \lcode{CUBIT}, which may contain tetrahedra, as well as hexahedra.
  As in case of \lcode{ASKI_type_inversion_grid = 2}, \ASKI output is stored at all inner GLL points of elements
  which are inside the volume defined by \lcode{ASKI_(cw)(xyz)}.
  \ASKI locates the wavefield points inside the inversion grid and computes weights.
\item (\lcode{TYPE_INVERSION_GRID = specfem3dInversionGrid}) \\
  Use \lcode{SPECFEM} elements as inversion grid:\\
  Wavefield points are \emph{all} GLL points of an element for elements which are (at least partly) inside the 
  volume defined by \lcode{ASKI_(cw)(xyz)}. Additionally store the jacobians for all wavefield points.
  Assume \lcode{ncell = ntot_wp/(NGLLX*NGLLY*NGLLY)} as the number of inversion grid cells, and the order of 
  wavefield points accordingly (\lcode{do k=1,NGLLZ;} \lcode{do j=1,NGLLY;} \lcode{do i=1,NGLLX;} \lcode{ip=ip+1 ....})
\end{enumerate}
%----------------------------------------------------------------------
\subsubsection*{\lcode{ASKI_(cw)(xyz), ASKI_rot_(XYZ)}}
%----------------------------------------------------------------------
Dependent on \lcode{ASKI_type_inversion_grid}, (a selection of) the following parameters may be used to define a volume 
within which wavefield points are searched for:

First, \lcode{ASKI_wx,ASKI_wy,ASKI_wz} define the total width of a block which is centered in \lcode{x=y=z=0}
E.g.\ the total block extension in x-direction covers all points with\\
\lcode{x >= - 0.5*ASKI_wx} and \lcode{x <=  0.5*ASKI_wx}.\\
Then, \lcode{ASKI_rot_X,ASKI_rot_Y,ASKI_rot_Z} define rotation angles in degrees by which the block is 
rotated (anti-clockwise) about the \lcode{Z}, \lcode{Y} and \lcode{X} coordinate axis, before 
\lcode{ASKI_cx,ASKI_cy,ASKI_cz} define a vector by which the rotated block is shifted (new center of block).

\emph{Be aware}:
\begin{itemize}
\item the parameters for rotation angles \lcode{ASKI_rot_(XYZ)} \emph{must always} be assinged to values! 
  Set to \lcode{0.} if not used.
\item \lcode{scartInversionGrid} only supports \lcode{ASKI_rot_Z} and uses a different definintion of the z-coverage.
\item \lcode{ecartInversionGrid} and \lcode{specfem3dInversionGrid} use \emph{all} rotation angles \lcode{ASKI_rot_(XYZ)}.
\end{itemize}
%
%-------------------------------
% SECTION install kernel output
%#############################################################
\section{Extend \lcodetitle{SPECFEM3D\_Cartesian-2.1} to produce output for \ASKI} \label{extent_to_ASKI}
%#############################################################
%
This section explains how to use the \lcode{SPECFEM3D_Cartesian} software 
\url{http://geodynamics.org/cig/software/specfem3d} as a forward method for \ASKI. In general, a regularly
installed \lcode{SPECFEM3D_Cartesian} version is extended by certain few modifications so it can produce
output for \ASKI. So, \lcode{SPECFEM3D_Cartesian for ASKI} basically has the same requirements and dependencies 
than the \lcode{SPECFEM3D_Cartesian} code, except that it needs a bit more memory and disc space. You should, 
therefore, have sufficient knowledge of how to run the regular \lcode{SPECFEM3D_Cartesian} software. Furthermore, 
you need an installation of \ASKI (obviously).
%
%----------------------------------------------------------------------
\subsection{Download and Dependencies} \label{extent_to_ASKI,sub:download_dependencies}
%----------------------------------------------------------------------
%
The extension package \lcode{SPECFEM3D_Cartesian-2.1 for ASKI} can be downloaded from 
\url{http://www.rub.de/ASKI}.

It assumes a running version of \lcode{SPECFEM3D_Cartesian} on your system which must be capable of a 
certain functionality. If the currently availabe release version of \lcode{SPECFEM3D_Cartesian} does not have 
this functionality, you can download \url{http://www.rub.de/ASKI}. 
This is a basic extract from the \lcode{SPECFEM3D_Cartesian} svn-repository (svn-revision by 9 june 2013, 
\lcode{SPECFEM3D} version \lcode{2.1}) which has been debugged and slightly modified such that it is capable of the 
required functionality. Important functionality is listed in the following, whereby the code references are related 
to the previously mentioned modified svn revision:
\begin{itemize}
\item in case of \lcode{USE_FORCE_POINT_SOURCE = .true.} in \lcode{Par_file} the sources should be interpolated inside the 
  source array and not just set to the closest GLL point (compare \lcode{src/specfem3D/locate_source.f90}, comments \lcode{FS FS})\\
  This, however, is not a strict necessity, you can also use \lcode{SPECFEM3D_Cartesian} codes which do not do this.
%
\item in case of \lcode{USE_FORCE_POINT_SOURCE = .true.} and \\
  \lcode{USE_RICKER_TIME_FUNCTION = .false.} in 
  \lcode{Par_file}, the regular Heaviside source time function (function \lcode{comp_source_time_function}) must
  be used (compare \lcode{src/specfem3D/compute_add_sources_viscoelastic.f90}, \lcode{src/specfem3D/setup_sources_receivers.f90} 
  comments \lcode{FS FS})
%
\item using external models (\lcode{MODEL = external} in \lcode{Par_file}), subroutine \lcode{model_external_values} must 
  be passed the default model values on enter (compare \lcode{src/generate_databases/get_model.f90}, comments \lcode{FS FS})
%
\item the general user interface of the \lcode{SPECFEM3D_Cartesian} code (i.e.\ definition of source mechanisms and receivers, 
  models, output files etc.) must be compatible with \lcode{SPECFEM3D_Cartesian} version \lcode{2.1} by june 2013.
\end{itemize}
%
%----------------------------------------------------------------------
\subsection{Installation}
%----------------------------------------------------------------------
%
Extract the files of the extension package \lcode{SPECFEM3D_Cartesian-2.1 for ASKI} into a subdirectory
of your \ASKI installation path, in the following refered to as \lcode{SPECFEM_for_ASKI}. I.e.\ the directory 
\lcode{SPECFEM_for_ASKI} is contained in directory \ASKI.\\
Install a \lcode{SPECFEM3D_Cartesian} code on your system which meets the requirements as in 
subsection~\ref{extent_to_ASKI,sub:download_dependencies}, in the following the \lcode{SPECFEM3D_Cartesian} 
installation path is refered to as \lcode{SPECFEM3D}.
\begin{itemize}
\item Follow the 9 items in the ``Installation'' section of file \lcode{ASKI/SPECFEM_for_ASKI/README}, whereby the first one
  should already be completed.
\item Adjust variables \lcode{SHELL}, \lcode{BLAS}, \lcode{LAPACK} etc.\ in \lcode{SPECFEM_for_ASKI/Makefile} in the 
  same way as you did in \lcode{ASKI/Makefile} for the installation of \ASKI. The same environment variables are assumed 
  here, too.
\item Run \lcode{make clean} and \lcode{make all} in \lcode{ASKI/SPECFEM_for_ASKI}.
\item Set \lcode{USE_SOURCES_RECVS_Z = .true.} in \lcode{SPECFEM3D/src/shared/constants.h} (or wherever your \lcode{constants.h}
  is).
\item Recompile your \lcode{SPECFEM3D_Cartesian} code by running \lcode{make clean} and \lcode{make all} in \lcode{SPECFEM3D}.
\end{itemize}
%
%-------------------------------
% SECTION GNU Free Documentation License
\newpage
% -*-LaTex-*-

%-----------------------------------------------------------------------------
%   The following content of the file is a verbatim copy of file 
%   http://www.gnu.org/licenses/fdl-1.3.tex which has been slightly modified 
%   (these changes do NOT effect the produced pdf document!): 
%     - adding the above lines as well as this comment block
%     - commenting the preamble of this LaTex file in order to embed its content
%       in the ASKI manual (main file: manual.tex) as a chapter
%     - renaming 
%       "\chapter*{\rlap{GNU Free Documentation License}}" in
%       "\section*{\rlap{GNU Free Documentation License}}"
%       as well as 
%       "\addcontentsline{toc}{chapter}{GNU Free Documentation License}"
%       "\addcontentsline{toc}{section}{GNU Free Documentation License}"
%       because the document class used does not provide chapters
%     - commenting LaTex commands "\addcontentsline{toc}{section}{...}" in order to
%       keep the table of contents of the main document (ASKI manual) more clean 
%       (only having one chapter entry "GNU Free Documentation License", but not 
%       having an entry for each section therein)
%   August 2016, Florian Schumacher
%-----------------------------------------------------------------------------
%
%% This is set up to run with pdflatex.
%%---------The file header---------------------------------------------
%\documentclass[a4paper,12pt]{book}
%
%\usepackage[english]{babel} %language selection
%\selectlanguage{english}
%
%\pagenumbering{arabic}
%
%\usepackage{hyperref}
%\hypersetup{colorlinks, 
%           citecolor=black,
%           filecolor=black,
%           linkcolor=black,
%           urlcolor=black,
%           bookmarksopen=true,
%           pdftex}
%
%\hfuzz = .6pt % avoid black boxes
%           
%\begin{document}
%%---------------------------------------------------------------------
\section*{\rlap{GNU Free Documentation License}}
\phantomsection  % so hyperref creates bookmarks
\addcontentsline{toc}{section}{GNU Free Documentation License}
%\label{label_fdl}

 \begin{center}

       Version 1.3, 3 November 2008


 Copyright \copyright{} 2000, 2001, 2002, 2007, 2008  Free Software Foundation, Inc.
 
 \bigskip
 
     \texttt{<http://fsf.org/>}
  
 \bigskip
 
 Everyone is permitted to copy and distribute verbatim copies
 of this license document, but changing it is not allowed.
\end{center}


\begin{center}
{\bf\large Preamble}
\end{center}

The purpose of this License is to make a manual, textbook, or other
functional and useful document ``free'' in the sense of freedom: to
assure everyone the effective freedom to copy and redistribute it,
with or without modifying it, either commercially or noncommercially.
Secondarily, this License preserves for the author and publisher a way
to get credit for their work, while not being considered responsible
for modifications made by others.

This License is a kind of ``copyleft'', which means that derivative
works of the document must themselves be free in the same sense.  It
complements the GNU General Public License, which is a copyleft
license designed for free software.

We have designed this License in order to use it for manuals for free
software, because free software needs free documentation: a free
program should come with manuals providing the same freedoms that the
software does.  But this License is not limited to software manuals;
it can be used for any textual work, regardless of subject matter or
whether it is published as a printed book.  We recommend this License
principally for works whose purpose is instruction or reference.


\begin{center}
{\Large\bf 1. APPLICABILITY AND DEFINITIONS\par}
\phantomsection
%\addcontentsline{toc}{section}{1. APPLICABILITY AND DEFINITIONS}
\end{center}

This License applies to any manual or other work, in any medium, that
contains a notice placed by the copyright holder saying it can be
distributed under the terms of this License.  Such a notice grants a
world-wide, royalty-free license, unlimited in duration, to use that
work under the conditions stated herein.  The ``\textbf{Document}'', below,
refers to any such manual or work.  Any member of the public is a
licensee, and is addressed as ``\textbf{you}''.  You accept the license if you
copy, modify or distribute the work in a way requiring permission
under copyright law.

A ``\textbf{Modified Version}'' of the Document means any work containing the
Document or a portion of it, either copied verbatim, or with
modifications and/or translated into another language.

A ``\textbf{Secondary Section}'' is a named appendix or a front-matter section of
the Document that deals exclusively with the relationship of the
publishers or authors of the Document to the Document's overall subject
(or to related matters) and contains nothing that could fall directly
within that overall subject.  (Thus, if the Document is in part a
textbook of mathematics, a Secondary Section may not explain any
mathematics.)  The relationship could be a matter of historical
connection with the subject or with related matters, or of legal,
commercial, philosophical, ethical or political position regarding
them.

The ``\textbf{Invariant Sections}'' are certain Secondary Sections whose titles
are designated, as being those of Invariant Sections, in the notice
that says that the Document is released under this License.  If a
section does not fit the above definition of Secondary then it is not
allowed to be designated as Invariant.  The Document may contain zero
Invariant Sections.  If the Document does not identify any Invariant
Sections then there are none.

The ``\textbf{Cover Texts}'' are certain short passages of text that are listed,
as Front-Cover Texts or Back-Cover Texts, in the notice that says that
the Document is released under this License.  A Front-Cover Text may
be at most 5 words, and a Back-Cover Text may be at most 25 words.

A ``\textbf{Transparent}'' copy of the Document means a machine-readable copy,
represented in a format whose specification is available to the
general public, that is suitable for revising the document
straightforwardly with generic text editors or (for images composed of
pixels) generic paint programs or (for drawings) some widely available
drawing editor, and that is suitable for input to text formatters or
for automatic translation to a variety of formats suitable for input
to text formatters.  A copy made in an otherwise Transparent file
format whose markup, or absence of markup, has been arranged to thwart
or discourage subsequent modification by readers is not Transparent.
An image format is not Transparent if used for any substantial amount
of text.  A copy that is not ``Transparent'' is called ``\textbf{Opaque}''.

Examples of suitable formats for Transparent copies include plain
ASCII without markup, Texinfo input format, LaTeX input format, SGML
or XML using a publicly available DTD, and standard-conforming simple
HTML, PostScript or PDF designed for human modification.  Examples of
transparent image formats include PNG, XCF and JPG.  Opaque formats
include proprietary formats that can be read and edited only by
proprietary word processors, SGML or XML for which the DTD and/or
processing tools are not generally available, and the
machine-generated HTML, PostScript or PDF produced by some word
processors for output purposes only.

The ``\textbf{Title Page}'' means, for a printed book, the title page itself,
plus such following pages as are needed to hold, legibly, the material
this License requires to appear in the title page.  For works in
formats which do not have any title page as such, ``Title Page'' means
the text near the most prominent appearance of the work's title,
preceding the beginning of the body of the text.

The ``\textbf{publisher}'' means any person or entity that distributes
copies of the Document to the public.

A section ``\textbf{Entitled XYZ}'' means a named subunit of the Document whose
title either is precisely XYZ or contains XYZ in parentheses following
text that translates XYZ in another language.  (Here XYZ stands for a
specific section name mentioned below, such as ``\textbf{Acknowledgements}'',
``\textbf{Dedications}'', ``\textbf{Endorsements}'', or ``\textbf{History}''.)  
To ``\textbf{Preserve the Title}''
of such a section when you modify the Document means that it remains a
section ``Entitled XYZ'' according to this definition.

The Document may include Warranty Disclaimers next to the notice which
states that this License applies to the Document.  These Warranty
Disclaimers are considered to be included by reference in this
License, but only as regards disclaiming warranties: any other
implication that these Warranty Disclaimers may have is void and has
no effect on the meaning of this License.


\begin{center}
{\Large\bf 2. VERBATIM COPYING\par}
\phantomsection
%\addcontentsline{toc}{section}{2. VERBATIM COPYING}
\end{center}

You may copy and distribute the Document in any medium, either
commercially or noncommercially, provided that this License, the
copyright notices, and the license notice saying this License applies
to the Document are reproduced in all copies, and that you add no other
conditions whatsoever to those of this License.  You may not use
technical measures to obstruct or control the reading or further
copying of the copies you make or distribute.  However, you may accept
compensation in exchange for copies.  If you distribute a large enough
number of copies you must also follow the conditions in section~3.

You may also lend copies, under the same conditions stated above, and
you may publicly display copies.


\begin{center}
{\Large\bf 3. COPYING IN QUANTITY\par}
\phantomsection
%\addcontentsline{toc}{section}{3. COPYING IN QUANTITY}
\end{center}


If you publish printed copies (or copies in media that commonly have
printed covers) of the Document, numbering more than 100, and the
Document's license notice requires Cover Texts, you must enclose the
copies in covers that carry, clearly and legibly, all these Cover
Texts: Front-Cover Texts on the front cover, and Back-Cover Texts on
the back cover.  Both covers must also clearly and legibly identify
you as the publisher of these copies.  The front cover must present
the full title with all words of the title equally prominent and
visible.  You may add other material on the covers in addition.
Copying with changes limited to the covers, as long as they preserve
the title of the Document and satisfy these conditions, can be treated
as verbatim copying in other respects.

If the required texts for either cover are too voluminous to fit
legibly, you should put the first ones listed (as many as fit
reasonably) on the actual cover, and continue the rest onto adjacent
pages.

If you publish or distribute Opaque copies of the Document numbering
more than 100, you must either include a machine-readable Transparent
copy along with each Opaque copy, or state in or with each Opaque copy
a computer-network location from which the general network-using
public has access to download using public-standard network protocols
a complete Transparent copy of the Document, free of added material.
If you use the latter option, you must take reasonably prudent steps,
when you begin distribution of Opaque copies in quantity, to ensure
that this Transparent copy will remain thus accessible at the stated
location until at least one year after the last time you distribute an
Opaque copy (directly or through your agents or retailers) of that
edition to the public.

It is requested, but not required, that you contact the authors of the
Document well before redistributing any large number of copies, to give
them a chance to provide you with an updated version of the Document.


\begin{center}
{\Large\bf 4. MODIFICATIONS\par}
\phantomsection
%\addcontentsline{toc}{section}{4. MODIFICATIONS}
\end{center}

You may copy and distribute a Modified Version of the Document under
the conditions of sections 2 and 3 above, provided that you release
the Modified Version under precisely this License, with the Modified
Version filling the role of the Document, thus licensing distribution
and modification of the Modified Version to whoever possesses a copy
of it.  In addition, you must do these things in the Modified Version:

\begin{itemize}
\item[A.] 
   Use in the Title Page (and on the covers, if any) a title distinct
   from that of the Document, and from those of previous versions
   (which should, if there were any, be listed in the History section
   of the Document).  You may use the same title as a previous version
   if the original publisher of that version gives permission.
   
\item[B.]
   List on the Title Page, as authors, one or more persons or entities
   responsible for authorship of the modifications in the Modified
   Version, together with at least five of the principal authors of the
   Document (all of its principal authors, if it has fewer than five),
   unless they release you from this requirement.
   
\item[C.]
   State on the Title page the name of the publisher of the
   Modified Version, as the publisher.
   
\item[D.]
   Preserve all the copyright notices of the Document.
   
\item[E.]
   Add an appropriate copyright notice for your modifications
   adjacent to the other copyright notices.
   
\item[F.]
   Include, immediately after the copyright notices, a license notice
   giving the public permission to use the Modified Version under the
   terms of this License, in the form shown in the Addendum below.
   
\item[G.]
   Preserve in that license notice the full lists of Invariant Sections
   and required Cover Texts given in the Document's license notice.
   
\item[H.]
   Include an unaltered copy of this License.
   
\item[I.]
   Preserve the section Entitled ``History'', Preserve its Title, and add
   to it an item stating at least the title, year, new authors, and
   publisher of the Modified Version as given on the Title Page.  If
   there is no section Entitled ``History'' in the Document, create one
   stating the title, year, authors, and publisher of the Document as
   given on its Title Page, then add an item describing the Modified
   Version as stated in the previous sentence.
   
\item[J.]
   Preserve the network location, if any, given in the Document for
   public access to a Transparent copy of the Document, and likewise
   the network locations given in the Document for previous versions
   it was based on.  These may be placed in the ``History'' section.
   You may omit a network location for a work that was published at
   least four years before the Document itself, or if the original
   publisher of the version it refers to gives permission.
   
\item[K.]
   For any section Entitled ``Acknowledgements'' or ``Dedications'',
   Preserve the Title of the section, and preserve in the section all
   the substance and tone of each of the contributor acknowledgements
   and/or dedications given therein.
   
\item[L.]
   Preserve all the Invariant Sections of the Document,
   unaltered in their text and in their titles.  Section numbers
   or the equivalent are not considered part of the section titles.
   
\item[M.]
   Delete any section Entitled ``Endorsements''.  Such a section
   may not be included in the Modified Version.
   
\item[N.]
   Do not retitle any existing section to be Entitled ``Endorsements''
   or to conflict in title with any Invariant Section.
   
\item[O.]
   Preserve any Warranty Disclaimers.
\end{itemize}

If the Modified Version includes new front-matter sections or
appendices that qualify as Secondary Sections and contain no material
copied from the Document, you may at your option designate some or all
of these sections as invariant.  To do this, add their titles to the
list of Invariant Sections in the Modified Version's license notice.
These titles must be distinct from any other section titles.

You may add a section Entitled ``Endorsements'', provided it contains
nothing but endorsements of your Modified Version by various
parties---for example, statements of peer review or that the text has
been approved by an organization as the authoritative definition of a
standard.

You may add a passage of up to five words as a Front-Cover Text, and a
passage of up to 25 words as a Back-Cover Text, to the end of the list
of Cover Texts in the Modified Version.  Only one passage of
Front-Cover Text and one of Back-Cover Text may be added by (or
through arrangements made by) any one entity.  If the Document already
includes a cover text for the same cover, previously added by you or
by arrangement made by the same entity you are acting on behalf of,
you may not add another; but you may replace the old one, on explicit
permission from the previous publisher that added the old one.

The author(s) and publisher(s) of the Document do not by this License
give permission to use their names for publicity for or to assert or
imply endorsement of any Modified Version.


\begin{center}
{\Large\bf 5. COMBINING DOCUMENTS\par}
\phantomsection
%\addcontentsline{toc}{section}{5. COMBINING DOCUMENTS}
\end{center}


You may combine the Document with other documents released under this
License, under the terms defined in section~4 above for modified
versions, provided that you include in the combination all of the
Invariant Sections of all of the original documents, unmodified, and
list them all as Invariant Sections of your combined work in its
license notice, and that you preserve all their Warranty Disclaimers.

The combined work need only contain one copy of this License, and
multiple identical Invariant Sections may be replaced with a single
copy.  If there are multiple Invariant Sections with the same name but
different contents, make the title of each such section unique by
adding at the end of it, in parentheses, the name of the original
author or publisher of that section if known, or else a unique number.
Make the same adjustment to the section titles in the list of
Invariant Sections in the license notice of the combined work.

In the combination, you must combine any sections Entitled ``History''
in the various original documents, forming one section Entitled
``History''; likewise combine any sections Entitled ``Acknowledgements'',
and any sections Entitled ``Dedications''.  You must delete all sections
Entitled ``Endorsements''.

\begin{center}
{\Large\bf 6. COLLECTIONS OF DOCUMENTS\par}
\phantomsection
%\addcontentsline{toc}{section}{6. COLLECTIONS OF DOCUMENTS}
\end{center}

You may make a collection consisting of the Document and other documents
released under this License, and replace the individual copies of this
License in the various documents with a single copy that is included in
the collection, provided that you follow the rules of this License for
verbatim copying of each of the documents in all other respects.

You may extract a single document from such a collection, and distribute
it individually under this License, provided you insert a copy of this
License into the extracted document, and follow this License in all
other respects regarding verbatim copying of that document.


\begin{center}
{\Large\bf 7. AGGREGATION WITH INDEPENDENT WORKS\par}
\phantomsection
%\addcontentsline{toc}{section}{7. AGGREGATION WITH INDEPENDENT WORKS}
\end{center}


A compilation of the Document or its derivatives with other separate
and independent documents or works, in or on a volume of a storage or
distribution medium, is called an ``aggregate'' if the copyright
resulting from the compilation is not used to limit the legal rights
of the compilation's users beyond what the individual works permit.
When the Document is included in an aggregate, this License does not
apply to the other works in the aggregate which are not themselves
derivative works of the Document.

If the Cover Text requirement of section~3 is applicable to these
copies of the Document, then if the Document is less than one half of
the entire aggregate, the Document's Cover Texts may be placed on
covers that bracket the Document within the aggregate, or the
electronic equivalent of covers if the Document is in electronic form.
Otherwise they must appear on printed covers that bracket the whole
aggregate.


\begin{center}
{\Large\bf 8. TRANSLATION\par}
\phantomsection
%\addcontentsline{toc}{section}{8. TRANSLATION}
\end{center}


Translation is considered a kind of modification, so you may
distribute translations of the Document under the terms of section~4.
Replacing Invariant Sections with translations requires special
permission from their copyright holders, but you may include
translations of some or all Invariant Sections in addition to the
original versions of these Invariant Sections.  You may include a
translation of this License, and all the license notices in the
Document, and any Warranty Disclaimers, provided that you also include
the original English version of this License and the original versions
of those notices and disclaimers.  In case of a disagreement between
the translation and the original version of this License or a notice
or disclaimer, the original version will prevail.

If a section in the Document is Entitled ``Acknowledgements'',
``Dedications'', or ``History'', the requirement (section~4) to Preserve
its Title (section~1) will typically require changing the actual
title.


\begin{center}
{\Large\bf 9. TERMINATION\par}
\phantomsection
%\addcontentsline{toc}{section}{9. TERMINATION}
\end{center}


You may not copy, modify, sublicense, or distribute the Document
except as expressly provided under this License.  Any attempt
otherwise to copy, modify, sublicense, or distribute it is void, and
will automatically terminate your rights under this License.

However, if you cease all violation of this License, then your license
from a particular copyright holder is reinstated (a) provisionally,
unless and until the copyright holder explicitly and finally
terminates your license, and (b) permanently, if the copyright holder
fails to notify you of the violation by some reasonable means prior to
60 days after the cessation.

Moreover, your license from a particular copyright holder is
reinstated permanently if the copyright holder notifies you of the
violation by some reasonable means, this is the first time you have
received notice of violation of this License (for any work) from that
copyright holder, and you cure the violation prior to 30 days after
your receipt of the notice.

Termination of your rights under this section does not terminate the
licenses of parties who have received copies or rights from you under
this License.  If your rights have been terminated and not permanently
reinstated, receipt of a copy of some or all of the same material does
not give you any rights to use it.


\begin{center}
{\Large\bf 10. FUTURE REVISIONS OF THIS LICENSE\par}
\phantomsection
%\addcontentsline{toc}{section}{10. FUTURE REVISIONS OF THIS LICENSE}
\end{center}


The Free Software Foundation may publish new, revised versions
of the GNU Free Documentation License from time to time.  Such new
versions will be similar in spirit to the present version, but may
differ in detail to address new problems or concerns.  See
\texttt{http://www.gnu.org/copyleft/}.

Each version of the License is given a distinguishing version number.
If the Document specifies that a particular numbered version of this
License ``or any later version'' applies to it, you have the option of
following the terms and conditions either of that specified version or
of any later version that has been published (not as a draft) by the
Free Software Foundation.  If the Document does not specify a version
number of this License, you may choose any version ever published (not
as a draft) by the Free Software Foundation.  If the Document
specifies that a proxy can decide which future versions of this
License can be used, that proxy's public statement of acceptance of a
version permanently authorizes you to choose that version for the
Document.


\begin{center}
{\Large\bf 11. RELICENSING\par}
\phantomsection
%\addcontentsline{toc}{section}{11. RELICENSING}
\end{center}


``Massive Multiauthor Collaboration Site'' (or ``MMC Site'') means any
World Wide Web server that publishes copyrightable works and also
provides prominent facilities for anybody to edit those works.  A
public wiki that anybody can edit is an example of such a server.  A
``Massive Multiauthor Collaboration'' (or ``MMC'') contained in the
site means any set of copyrightable works thus published on the MMC
site.

``CC-BY-SA'' means the Creative Commons Attribution-Share Alike 3.0
license published by Creative Commons Corporation, a not-for-profit
corporation with a principal place of business in San Francisco,
California, as well as future copyleft versions of that license
published by that same organization.

``Incorporate'' means to publish or republish a Document, in whole or
in part, as part of another Document.

An MMC is ``eligible for relicensing'' if it is licensed under this
License, and if all works that were first published under this License
somewhere other than this MMC, and subsequently incorporated in whole
or in part into the MMC, (1) had no cover texts or invariant sections,
and (2) were thus incorporated prior to November 1, 2008.

The operator of an MMC Site may republish an MMC contained in the site
under CC-BY-SA on the same site at any time before August 1, 2009,
provided the MMC is eligible for relicensing.


\begin{center}
{\Large\bf ADDENDUM: How to use this License for your documents\par}
\phantomsection
%\addcontentsline{toc}{section}{ADDENDUM: How to use this License for your documents}
\end{center}

To use this License in a document you have written, include a copy of
the License in the document and put the following copyright and
license notices just after the title page:

\bigskip
\begin{quote}
    Copyright \copyright{}  YEAR  YOUR NAME.
    Permission is granted to copy, distribute and/or modify this document
    under the terms of the GNU Free Documentation License, Version 1.3
    or any later version published by the Free Software Foundation;
    with no Invariant Sections, no Front-Cover Texts, and no Back-Cover Texts.
    A copy of the license is included in the section entitled ``GNU
    Free Documentation License''.
\end{quote}
\bigskip
    
If you have Invariant Sections, Front-Cover Texts and Back-Cover Texts,
replace the ``with \dots\ Texts.''\ line with this:

\bigskip
\begin{quote}
    with the Invariant Sections being LIST THEIR TITLES, with the
    Front-Cover Texts being LIST, and with the Back-Cover Texts being LIST.
\end{quote}
\bigskip
    
If you have Invariant Sections without Cover Texts, or some other
combination of the three, merge those two alternatives to suit the
situation.

If your document contains nontrivial examples of program code, we
recommend releasing these examples in parallel under your choice of
free software license, such as the GNU General Public License,
to permit their use in free software.

%%---------------------------------------------------------------------
%\end{document}

%

\end{document}
